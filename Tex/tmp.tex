\documentclass[hidelinks,12pt]{article}
\usepackage[utf8]{inputenc}
\usepackage[T1]{fontenc}

%% A SUPPRIMER A LA FIN
\usepackage[french]{babel}

\usepackage{times}
\usepackage{hyperref}
\usepackage{graphicx}
\usepackage{caption,graphicx,enumitem}
\usepackage{amsmath}
%\usepackage{titlesec}

%\setcounter{secnumdepth}{4}


\begin{document}

\title{Study of a surrogate model for shallow water equations}
\author{Bastien Nony, Alban Gossard\\
Institut National des Sciences Appliquées,\\
Toulouse,\\
\href{mailto:nony@etud.insa-toulouse.fr}{   \texttt{nony@etud.insa-toulouse.fr}}\\
\href{mailto:gossard@etud.insa-toulouse.fr}{   \texttt{gossard@etud.insa-toulouse.fr}}}
\date{\today}

\maketitle

\begin{abstract}
In this work we study surrogates problems for different types of modelling problem. The objective is to provide fast calculation for undetermined values. Beginning from physical equations such as Saint-Venant's, we add statistical formulas to determine the variability of the system. This study registers in the frame of geostatistics.
\end{abstract}

\newpage

\tableofcontents











\section{Objectives}
\section{Key notions}


\section{Sources d’incertitude}

%ATTENTION : paragraphes à approfondir car une seule source (Fajraoui_noura2014)


\subsection{Incertitude structurelle }

Incertitude liée au modèle mathématique, approximation de la réalité. Les hypothèses simplifient les phénomènes physiques et/ou ne les prennent pas tous en considération. Dans notre cas les équations de Saint Venant sont des équations 1D en eau peu profonde.

\subsection{Incertitude numérique}


Imprécisions liées aux approximations numériques qui s’accumulent dans les calculs. les solutions ne sont pas forcément analytiques et chaque étape de calcul peut conduire à une approximation qui s’additionne aux précédentes, par exemple enchaînement d’erreurs d’arrondi. L’incertitude numérique concerne aussi l’erreur de discrétisation spatiale obtenue lors des relevés physiques effectués par les spécialistes sur le terrain.

\subsection{Incertitude paramétrique}

Incertitude obtenue par la variabilité des paramètres d’entrée ou du manque d’information ou du biais d’un échantillon de départ. Un test statistique effectué sur un échantillon non représentatif peut mener à des résultats complètement biaisés.





Introduisons quelques outils statistiques :

\subsection{Variogramme}


\subsection{Interpolation par krigeage}

Cette méthode est probablement la plus évidente relativement au formalisme mathématique. Il s'agit d'une estimation linéaire où chaque estimation se fera linéairement en fonction de valeurs exactes mesurées au environs et associées à des poids particuliers.

Ainsi, la procédure est la suivante :

(1) On délimite notre zone d'étude en intervalles de paramétrage.

(2) On mesure, calcule à partir d'un modèle physique de départs un certain nombre de données idéales ($X_1,\ldots,X_n$)

(3) Pour un certain point donné dans notre espace délimité on trouve l'estimateur associé en calculant la matrice de poids correspondante.
(4) notre estimateur en ce point vaut donc la somme des valeurs idéales * leur poids estimé correspondant

Mathématiquement cela donne :

(1) pour un krigeage simple

(2) Pour un krigeage ordinaire

Cette approche apporte plusieurs avantages :

Tout d'abord il est possible de quantifier la variance de notre estimation par le biais de la variance du krigeage :

Les poids de krigeage adoptent des caractéristiques logiques :

(1)A l'infini les points n'apportent plus d'information sur le résultat

(2) Si le nombre de valeurs dans une région donnée est grand alors les poids sont très faibles. Cela s'explique par le fait que chaque point influe davantage sur les zones qui lui sont très proches et moins sur les zones lointaines où l'information se partage entre beaucoup de données.

(3) Dans les régions où il y a peu de données, le krigeage reflète une estimation de la moyenne.

\section{Méthode de quadrature}
\section{Interpolation par polynomes du chaos}

Comment choisir les points de référence? Méthode quasi aléatoires
\section{Méthodes d’échantillonage}
Dans ce paragraphe nous traitons la problématique de l’échantillonnage de l’espace de départ. L’échantillon de départ est limité à N données. Comment les choisir de manière optimale? 

Une première idée consiste à les quadriller de manière régulière notre espace de départ. Pour des raisons de périodicité et de régularité cette méthode peut conduire à l’obtention d’un échantillon non représentatif de la population de départ. Par exemple nous voulons modéliser la fonction sinus sur l’intervalle $[0,2k\pi ]$. Si les relevés sont effectués régulièrement tous les $2i\pi$ alors on pourra penser que la fonction sinus est constante égale à 1. 

D’autres approches permettent de limiter ce risque. Nous allons tout d’abord voir les méthodes de Monte-Carlo puis des méthodes quasi-aléatoires comme la méthode de séquencement de Halton.

\subsection{Méthodes de Monte-Carlo}
Les échantillons ont tendance à se regrouper autour de la région à forte densité


\subsection{Latin Hypercube Sampling}
Meilleur couverture du sampling

\subsection{Séquence de halton}

















\end{document}
